\documentclass[letterpaper, 12pt]{article}
\usepackage[margin=1in]{geometry}
\usepackage{graphicx}
\graphicspath{{Figures/}{./}}
\usepackage{apacite}
\usepackage{amsmath}
\usepackage{amssymb}
\usepackage{amsthm}
\usepackage{indentfirst}
\usepackage{siunitx}
\usepackage[justification=centering]{caption}
\usepackage{float}

\title{PHYSICS SL
\\
Barton's Pendulum}
\author{}
\date{}

\begin{document}
\nocite{*}

\maketitle
\begin{center}
    Candidate Code:
    \\
    Session: May 2024
    \\
    Page Count:
\end{center}
\newpage

\tableofcontents
\newpage

\section{Research Question}

What is the relationship between the amplitude of
the forced pendulum /\unit{rad} and the
length of the driver pendulum /\unit{m} in a Barton's pendulum.

\section{Introduction}

Pendulums undergo a repeating cycle of energy transfer from
only potential energy to only kinetic energy back to
only potential energy. This process causes
a pendulum system to undergo simple harmonic motion,
therefore pendulums have properties such as
period and frequency of oscillation defined
that are dependent on the length of the pendulum.

What this also means is that any periodic
external force will or will not resonate with
the pendulum. This applies to various real-life
scenarios and problems, from something as little
as the frequency that a parent should push
their child on a swing to whether gusts
of wind are capable of driving an idle wrecking ball
to dangerous.

Intuitively, if an external periodic force
is resonant with the pendulum, then the extent
that the pendulum's amplitude will reach
will be at its maximum. However, one should question exactly
how this maximum amplitude grows as the frequency
of the external force approaches the resonant frequency
of the pendulum.

This could easily be investigated by suspending
a driving pendulum and a forced pendulum on
the same string. This allows for an easy way
to provide an external periodic force through
the driver pendulum, allowing for the frequency
of the external force to be manipulated through
changing the length of the driver pendulum and
removing the necessity of a motorized
instrument.

\section{Background information}



\section{Hypothesis}

When the length of the driver pendulum equals to the length of the forced pendulum,
then the amplitude of the forced pendulum will be at its maximum as the two pendulums
are in resonance.

As the length of the driver pendulum approaches the length of the forced pendulum,
then the amplitude of the forced pendulum will approach that maximum from lower values.

\section{Materials}

\section{Experimental Protocol}

\section{Variables}

\section{Raw Data}

\section{Processed Data}

\section{Analysis}

\section{Evaluation}

\bibliographystyle{apacite}
\bibliography{IB_PHYSICS_IA.bib}

\end{document}